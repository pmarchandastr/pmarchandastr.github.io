%%%%%%%%%%%%%%%%%%%%%%%%%%%%%%%%%%%%%%%%%
% "ModernCV" CV and Cover Letter
% LaTeX Template
% Version 1.2 (25/3/16)
%
% This template has been downloaded from:
% http://www.LaTeXTemplates.com
%
% Original author:
% Xavier Danaux (xdanaux@gmail.com) with modifications by:
% Vel (vel@latextemplates.com)
%
% License:
% CC BY-NC-SA 3.0 (http://creativecommons.org/licenses/by-nc-sa/3.0/)
%
% Important note:
% This template requires the moderncv.cls and .sty files to be in the same 
% directory as this .tex file. These files provide the resume style and themes 
% used for structuring the document.
%
%%%%%%%%%%%%%%%%%%%%%%%%%%%%%%%%%%%%%%%%%

%----------------------------------------------------------------------------------------
%	PACKAGES AND OTHER DOCUMENT CONFIGURATIONS
%----------------------------------------------------------------------------------------

\documentclass[11pt,a4paper,sans]{moderncv} % Font sizes: 10, 11, or 12; paper sizes: a4paper, letterpaper, a5paper, legalpaper, executivepaper or landscape; font families: sans or roman

\moderncvstyle{casual} % CV theme - options include: 'casual' (default), 'classic', 'oldstyle' and 'banking'
\moderncvcolor{red} % CV color - options include: 'blue' (default), 'orange', 'green', 'red', 'purple', 'grey' and 'black'

\usepackage{lipsum} % Used for inserting dummy 'Lorem ipsum' text into the template

\usepackage[scale=0.75]{geometry} % Reduce document margins
%\setlength{\hintscolumnwidth}{3cm} % Uncomment to change the width of the dates column
%\setlength{\makecvtitlenamewidth}{10cm} % For the 'classic' style, uncomment to adjust the width of the space allocated to your name

%----------------------------------------------------------------------------------------
%	NAME AND CONTACT INFORMATION SECTION
%----------------------------------------------------------------------------------------

\firstname{Pierre} % Your first name
\familyname{Marchand} % Your last name

% All information in this block is optional, comment out any lines you don't need
\title{Postdoc researcher in astrophysics}
\address{08/10/1990}{French -- 1 child (2016)}
\mobile{+33 6 30 21 82 60}
\email{pierre.marchand.astr@gmail.com}
\homepage{http://pmarchandastr.github.io}{http://pmarchandastr.github.io}
\photo[70pt][0.4pt]{photodemoi.png} % The first bracket is the picture height, the second is the thickness of the frame around the picture (0pt for no frame)

%----------------------------------------------------------------------------------------

\begin{document}
%----------------------------------------------------------------------------------------
%	CURRICULUM VITAE
%----------------------------------------------------------------------------------------

\makecvtitle % Print the CV title

\vspace{-1cm}

\section{Experience}

\subsection{Research}{

\cventry{Since Dec 2019}{Postdoc}{American Museum of Natural History}{New-York, USA}{}{Star formation simulations with non-ideal MHD, grain physics and chemistry}
\cventry{Oct 2017 -- Oct 2019}{Postdoc}{Osaka University}{Osaka, Japan}{JSPS Fellowship}{Star formation simulations with non-ideal MHD.}
\cventry{Oct 2014 -- Sep 2017}{PhD Thesis}{CRAL}{Lyon, France}{}{Study of physical processes involved in star formation.
\begin{itemize}
\item Numerical simulations (Fortran 90, AMR, Parallel Computing),
\item Non-ideal Magnetohydrodynamics,
\item Chemistry.
\end{itemize}
}

\cventry{Mar-Jul 2013}{Internship}{CRAL}{Lyon, France}{}{Study of the pulsations of gaseous planets.}

\cventry{May-Jul 2012}{Internship}{Tohoku University}{Sendai, Japan}{}{Study of the light profiles of distant galaxies from observational data.}

%\cventry{Jul 2011}{Internship}{Savoie R\'efractaires}{V\'enissieux, France}{}{Materials handling of refractory bricks.}




\subsection{Teaching}

\cventry{Jul. 2020}{Supervision of internships}{American Museum of Natural History}{New-York, USA}{}{6 weeks of research internships for future Earth Science Teachers.}
\cventry{2015 -- 2017}{Teaching assistant in mathematics}{Universit\'e Claude Bernard Lyon 1}{Lyon, France}{}{Tutorial classes and oral examinations in first and second year of University (192h).}
\cventry{2015}{Teaching assistant in sustainable development}{Universit\'e Claude Bernard Lyon 1}{Lyon, France}{}{Short lectures on energy sources for second year students (7x90min).}
\cventry{2013 -- 2017}{Private tuitions}{Methodia}{Lyon \& Paris, France}{}{Private tuitions for students in mathematics and physics from Middle School to 3rd year of University. Training classes in mathematics to prepare competitive examinations.}



\section{Publications and communications}

\subsection{Published papers}
\cvitem{2020}{Guillet V., Hennebelle P., Pineau des for\^ets G., Marcowith A., Commer\c{c}on B., \textbf{Marchand P.} 2020, A\&A (in press) : \textit{Dust coagulation feedback on magnetohydrodynamic resistivities in protostellar collapse}.}
\cvitem{2020}{\textbf{Marchand P.}, Tomida K., Tanaka K.E.I., Commer\c{c}on B., Chabrier G. 2019, ApJ (in press) : \textit{Protostellar collapse: regulation of the angular momentum and onset of an ionic precursor.}.}
\cvitem{2019}{\textbf{Marchand P}., Tomida K.,Commer\c{c}on B., Chabrier G. 2018, A\&A, 631, A66 : \textit{Impact of the Hall effect in star formation, improving the angular momentum conservation}.}
\cvitem{2018}{\textbf{Marchand P.},Commer\c{c}on B., Chabrier G. 2018, A\&A, 619, A37 : \textit{Impact of the Hall effect in star formation and the issue of angular momentum conservation}.}
\cvitem{2016}{Hennebelle P.,Commer\c{c}on B., Chabrier G. \& \textbf{Marchand P.} 2016, ApJ, 830L, 8H : \textit{Magnetically Self-regulated Formation of Early Protoplanetary Disks}.}
\cvitem{2016}{\textbf{Marchand P.}, Masson J., Chabrier G., Hennebelle P.,Commer\c con B., \& Vaytet N. 2016, A\&A, 592, A18 : \textit{Chemical solver to compute molecule and grain abundances and non-ideal MHD resistivities in prestellar core-collapse calculations}.}


\subsection{Conferences and seminars}
\cvitem{Dec 2019}{Seminar at American Museum of Natural History, New-York, USA: \textit{Chemistry and regulation of the angular momentum in star formation}.}
\cvitem{Sep 2019}{Seminar at Center for Computational Astrophysics (CCA), New-York, USA: \textit{Regulating the angular momentum in star formation}.}
\cvitem{Sep 2019}{Oral presentation at "GothamFest", New-York, USA: \textit{Regulating the angular momentum in star formation}.}
\cvitem{Aug 2019}{Seminar at Kagoshima University, Kagoshima, Japan: \textit{Regulating the angular momentum in star formation}.}
\cvitem{Mar 2019}{Seminar at University of Western Ontario, London, Canada: \textit{Chemistry and the Hall effect in star formation}.}
\cvitem{Mar 2019}{Seminar at Princeton University, Princeton, USA: \textit{Chemistry and the Hall effect in star formation}.}
\cvitem{Nov 2018}{Seminar at Tokyo University, Tokyo, Japan: \textit{Chemistry and non-ideal MHD for star formation}.}
\cvitem{Sep 2018}{Seminar at IPAG, Grenoble, France: \textit{The Hall effect in star formation}.}
\cvitem{Sep 2018}{Oral presentation at the RAMSES User Meeting, Lyon, France: \textit{The Hall effect in RAMSES for star formation}.}
\cvitem{Mar 2018}{Seminar at Kyushu University, Fukuoka, Japan: \textit{The Hall effect in star formation}.}
\cvitem{Jun 2016}{Oral presentation at the Astronomy and Astrophysics French Society (SF2A), stellar physics session (PNPS), Lyon, France: \textit{Chemistry and non-ideal MHD in star formation}.}
\cvitem{Sep 2015}{Oral presentation at the RAMSES User Meeting, Oxford, United Kingdom: \textit{The Hall effect in RAMSES}.}
\cvitem{Jun 2015}{Poster session at the \textit{Disc Dynamics \& Planet Formation} conference, Lanarka, Cyprus: \textit{Chemistry for non-ideal MHD}.}


\subsection{Outreach}
\cvitem{Jun 2020}{Participation to an astronomy panel for high school students for the end of their rocket project, New-York, USA.}
\cvitem{Feb 2020}{Scientific consultant for the French translation of the new planetarium show of the American Museum of Natural History "Worlds Beyond Earth".}
\cvitem{Jun 2019}{Conference on star formation for high school students, Takamatsu, Japan.}
\cvitem{Dec 2018}{Conference on the life cycle of stars for 2nd year scientific high school students, Kawanishi, Japan.}
\cvitem{Apr 2015, 2016, 2017}{"Astro week", Accompanying high school students to visit the planetarium, computing center and Lyon observatory, France.}
\cvitem{2015 -- 2017}{Presentation of the astrophysics department at ENS of Lyon and astronomy-related subjects for visiting middle and high school students ($\sim$10x90 minutes).}
\cvitem{May 2016}{Presentation of astronomy subjects for 10 years old elementary school children (90 minutes).}
\cvitem{Mar 2012}{"Nuit de l'equinoxe", Stand at a public gathering of the astronomy clubs of Lyon.}


\section{Education}

\cventry{Sep 2013 -- Oct 2014}{Master Degree in Astronomy and Astrophysics}{Paris observatory}{}{}{Theoretical astrophysics (fluid mechanics, radiative transfer, computational astrophysics). Ranking: 7th/33.}  % Arguments not required can be left empty
\cventry{Sep 2010 -- Sep 2013}{Engineering degree}{\'Ecole Centrale}{Lyon}{}{Specialised in Energy (electrical network, nuclear energy).}
\cventry{Sep 2008 -- Jun 2010}{Classes pr\'eparatoires}{Lyc\'ee Chaptal}{Paris}{}{Intensive courses of mathematics and physics to prepare competitive examinations. Accepted in \'Ecole Centrale de Lyon.}

%\subsection{School projects}
%
%\cvitem{2012}{Theoretical and numerical study of a photon trap.}
%\cvitem{2011}{Design and building of an aluminium crafting oven for Senegalees craftsmen.}



\section{Computer skills}

\cvitem{Advanced}{Fortran 90, \LaTeX}
\cvitem{Intermediate}{\textsc{python}, Office, Linux, Microsoft Windows}
\cvitem{Basic}{Parallel Computing, \textsc{html}, CSS, C++, Matlab, \textsc{maple}, Blender}


\section{Languages}

\cvitemwithcomment{French}{Mothertongue}{}
\cvitemwithcomment{English}{Fluent}{}
\cvitemwithcomment{Japanese}{Conversational}{}
\cvitemwithcomment{Spanish}{Basic}{}


%\section{Interests}
%
%\renewcommand{\listitemsymbol}{-~} % Changes the symbol used for lists
%
%\cvlistitem{Astronomy (President of the Astronomy club of the \'Ecole Centrale de Lyon in 2011)}
%\cvlistdoubleitem{Writing}{Reading (science fiction)}
%\cvlistdoubleitem{Video games}{Japan}
%\cvlistdoubleitem{History}{Theatrical improvisation}

%----------------------------------------------------------------------------------------

\end{document}
